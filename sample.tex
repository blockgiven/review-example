\chapter{はじめてのRe:VIEW}
\label{chap:sample}

\begin{quotation}
「Hello, Re:VIEW.」
\end{quotation}

\section{Re:VIEWとは}
\label{sec:1-1}

\textbf{Re:VIEW}は、EWBやRDあるいはWikiに似た簡易フォーマットで記述したテキストファイルを、目的に応じて各種の形式に変換するツールセットです。

平易な文法ながらも、コンピュータ関係のドキュメント作成のための多くの機能を備えており、次のような形式に変換できます。

\begin{itemize}
\item テキスト(指示タグ付き)
\item LaTeX
\item HTML
\item XML
\end{itemize}

現在入手手段としては次の3つがあります。

\begin{enumerate}
\item Ruby gem
\item Git
\item Subversion
\end{enumerate}

ホームページは\texttt{https://github.com/kmuto/review/wiki/}です。
